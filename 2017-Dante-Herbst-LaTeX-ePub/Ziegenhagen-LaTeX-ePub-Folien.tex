\documentclass[12pt,ngerman]{beamer}

\usepackage[utf8]{inputenc}
\usepackage[T1]{fontenc}
\usepackage{booktabs}
\usepackage{babel}
\usepackage{graphicx}
\usepackage{csquotes}
\usepackage{xcolor}

\author{Uwe Ziegenhagen}
\title{EPUBs erzeugen mit LaTeX, lwarp und Calibre}

\begin{document}

\begin{frame}

\maketitle

\end{frame}

\begin{frame}
\frametitle{Inhalt}

\tableofcontents

\end{frame}

\section{Was ist eigentlich \enquote{EPUB}?}

\begin{frame}
\frametitle{Was ist eigentlich \enquote{EPUB}?}
\framesubtitle{Einführung}

EPUB

\begin{itemize}
\item Akronym für \enquote{electronic publication}
\item offener Standard für E-Books 
\item International Digital Publishing Forum (IDPF)
\item ersetzt Open eBook (OEB/OEBPS) Standard
\item dynamische Anpassung des Textes an Ausgabegerät
\item kein Gerät implementiert Standard komplett
\item DRM optional
\end{itemize}

\end{frame}

\begin{frame}
\frametitle{Was ist eigentlich \enquote{EPUB}?}
\framesubtitle{Der EPUB Standard}

\textbf{Version 2}

\begin{itemize}
\item XML und XHTML,
\item DTBook, NCX, SVG, CSS,
\item Dublin Core und Zip,
\item sowie PNG, JPEG/JFIF, GIF
\end{itemize}

\textbf{Version 3}

\begin{itemize}
	\item ohne DTBook
	\item Medienüberlagerungen ($\Rightarrow$ Hörbuch)
	\item \enquote{Kanonische Fragmentidentifizierer} $\Rightarrow$ genaue Verweise
	\item Präsentation: seitenbasiert oder rollbar
\end{itemize}

\end{frame}


\begin{frame}
\frametitle{Was ist eigentlich \enquote{EPUB}?}
\framesubtitle{Aufbau von EPUB-Dateien}

\begin{itemize}
\item Siehe Beispiel von \url{http://www.inkshard.com/how-to-make-an-ebook-epub-file/}
\item Validator unter \url{http://validator.idpf.org/}
\item 
\item 
\item 
\item 
\end{itemize}
\end{frame}

\section{Das \enquote{lwarp}-Paket}

\begin{frame}
\frametitle{Das \enquote{lwarp}-Paket}
\framesubtitle{~}

\begin{itemize}
\item Autor: 
\item Sehr aktiv in der Entwicklung, seit xx über 40 Versionen
\item $\Rightarrow$ \TeX-Installation vorher updaten
\item 
\item 
\item 
\end{itemize}
\end{frame}

\begin{frame}
\frametitle{Das \enquote{lwarp}-Beispiel}
\framesubtitle{~}

\begin{itemize}
\item Beispiel aus dem Handbuch
\item 
\item 
\item 
\item 
\item 
\end{itemize}
\end{frame}


\begin{frame}
\frametitle{Calibre}
\framesubtitle{~}

\begin{itemize}
\item Was ist Calibre?
\item 
\item 
\item 
\item 
\item 
\end{itemize}
\end{frame}




\end{document}