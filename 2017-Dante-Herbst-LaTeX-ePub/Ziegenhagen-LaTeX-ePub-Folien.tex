%!TEX TS-program = Arara
% arara: pdflatex
\documentclass[12pt,ngerman]{beamer}

\usepackage[utf8]{inputenc}
\usepackage[T1]{fontenc}
\usepackage{booktabs}
\usepackage{babel}
\usepackage{graphicx}
\usepackage{csquotes}
\usepackage{xcolor}

\author{Uwe Ziegenhagen}
\title{EPUBs erzeugen mit LaTeX, lwarp und Calibre}

\begin{document}

\begin{frame}

\maketitle

\end{frame}

\begin{frame}
\frametitle{Inhalt}

\tableofcontents

\end{frame}

\section{Was ist eigentlich \enquote{ePub}?}

\begin{frame}
\frametitle{Was ist eigentlich \enquote{ePub}?}
\framesubtitle{~}

Verschiedene eBook-Formate gebräuchlich

\begin{itemize}
\item PDF
\item Kindle
\item ePub
\item 
\item 
\item 
\end{itemize}

\end{frame}

\begin{frame}
\frametitle{Aufbau von ePub-Dateien}
\framesubtitle{~}

\begin{itemize}
\item Verschiedene Versionen
\item 
\item 
\item 
\item 
\item 
\end{itemize}
\end{frame}

\section{Das \enquote{lwarp}-Paket}

\begin{frame}
\frametitle{Das \enquote{lwarp}-Paket}
\framesubtitle{~}

\begin{itemize}
\item Autor: 
\item Sehr aktiv in der Entwicklung, seit xx über 40 Versionen
\item $\Rightarrow$ \TeX-Installation vorher updaten
\item 
\item 
\item 
\end{itemize}
\end{frame}

\begin{frame}
\frametitle{Das \enquote{lwarp}-Beispiel}
\framesubtitle{~}

\begin{itemize}
\item Beispiel aus dem Handbuch
\item 
\item 
\item 
\item 
\item 
\end{itemize}
\end{frame}


\begin{frame}
\frametitle{Calibre}
\framesubtitle{~}

\begin{itemize}
\item Was ist Calibre?
\item 
\item 
\item 
\item 
\item 
\end{itemize}
\end{frame}




\end{document}