% Save this as tutorial.tex for the lwarp package tutorial.
% Example from the lwarp documentation
\documentclass{book}
\usepackage{iftex}
% --- LOAD FONT SELECTION AND ENCODING BEFORE LOADING LWARP ---
\ifPDFTeX
\usepackage{lmodern} % pdflatex
\usepackage[T1]{fontenc}
\usepackage[utf8]{inputenc}
\else
\usepackage{fontspec} % XeLaTeX or LuaLaTeX
\fi
% --- LWARP IS LOADED NEXT ---
\usepackage[
% HomeHTMLFilename=index, % Filename of the homepage.
% HTMLFilename={node-}, % Filename prefix of other pages.
% IndexLanguage=english, % Language for xindy index, glossary.
% latexmk, % Use latexmk to compile.
% OSWindows, % Force Windows. (Usually automatic.)
% mathjax, % Use MathJax to display math.
]{lwarp}

\booltrue{FormatEPUB}

% \boolfalse{FileSectionNames} % If false, numbers the files.
% --- OTHER PACKAGES ARE LOADED AFTER LWARP ---
\usepackage{makeidx} \makeindex
\usepackage{xcolor} % (Demonstration purposes only.)
\usepackage{hyperref,cleveref} % LOAD THESE LAST!
% --- LATEX AND HTML CUSTOMIZATION ---
\title{The Lwarp Tutorial}
\author{Some Author}
\setcounter{tocdepth}{2} % Include subsections in the \TOC.
\setcounter{secnumdepth}{2} % Number down to subsections.
\setcounter{FileDepth}{1} % Split \HTML\ files at sections
\booltrue{CombineHigherDepths} % Combine parts/chapters/sections
\setcounter{SideTOCDepth}{1} % Include subsections in the side\TOC
\HTMLAuthor{Some Author} % Sets the HTML meta author tag.
\HTMLLanguage{en-US} % Sets the HTML meta language.
\HTMLDescription{A description.}% Sets the HTML meta description.
\HTMLFirstPageTop{Name and \fbox{HOMEPAGE LOGO}}

\HTMLPageTop{\fbox{LOGO}}
\HTMLPageBottom{Contact Information and Copyright}
\CSSFilename{lwarp_sagebrush.css}
\begin{document}
\maketitle % Or titlepage/titlingpage environment.
% An article abstract would go here.
\tableofcontents % MUST BE BEFORE THE FIRST SECTION BREAK!
\listoffigures
\chapter{First chapter}
\section{A section}
This is some text which is indexed.\index{Some text.}
\subsection{A subsection}
See \cref{fig:withtext}.
\begin{figure}\begin{center}
\fbox{\textcolor{blue!50!green}{Text in a figure.}}
\caption{A figure with text\label{fig:withtext}}
\end{center}\end{figure}
\section{Some math}
Inline math: $r = r_0 + vt - \frac{1}{2}at^2$
followed by display math:
\begin{equation}
a^2 + b^2 = c^2
\end{equation}
\begin{warpprint} % For print output ...
\cleardoublepage % ... a common method to place index entry into TOC.
\phantomsection
\addcontentsline{toc}{chapter}{\indexname}
\end{warpprint}
\ForceHTMLPage % HTML index will be on its own page.
\ForceHTMLTOC % HTML index will have its own toc entry.
\printindex
\end{document}